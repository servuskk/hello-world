%用来打印中文
\documentclass[UTF8]{ctexart}
\RequirePackage{iftex}
\RequirePackage{fix-cm}
\RequirePackage{fixltx2e}
%页面设置
\RequirePackage{geometry}
\if@twoside
 \geometry{a4paper,
 bindingoffset = 2cm,
 inner = 0.5cm,
 outer = 2cm,
 top = 3cm,
 bottom = 2cm
 }
\else
 \geometry{a4paper,
 left = 2cm,
 right = 2cm,
 top = 2cm,
 bottom = 2cm
 }
%各种包
\usepackage{fancyhdr}
\usepackage{amssymb}
\RequirePackage{graphicx}
\RequirePackage{subfigure}
\RequirePackage{caption}
\RequirePackage{diagbox}
\RequirePackage{multirow}
\RequirePackage{makecell}
\RequirePackage{booktabs}
%\usepackage{lipsum}
\usepackage{mathtools}
\usepackage{listings}%代码
%矩阵
\usepackage{amsmath,xcolor}
\RequirePackage{longtable}
\RequirePackage{array}
%页眉
\RequirePackage{float}
\RequirePackage{flowchart}
\pagestyle{fancy}
\renewcommand{\headrulewidth}{0.5pt} 
\lhead{} \rhead{whh\ allesgutewh@gmail.com}%\thepage
% 顶格写\noindent
\begin{document}
%\pagenumbering{arabic}
\section*{Information Theory Homework-ch7}
%姓名(学号)第x章作业.pdf
\begin{center}
\today
\end{center}
\subsection*{Assignments}


7.1, 7.2, 7.5, 7.13, 7.18, 7.20, 7.22, 7.28(a)(c), 7.29, 7.32, 7.33 


\subsection*{7.1•输出的预处理}
(a)$X\rightarrow Y\rightarrow g(Y) $构成马尔科夫链,则$I(X;Y)\geqslant I(X;g(Y))$

$C=max_{p(x)}I(X;Y)\geqslant max_{p(x)}I(X;g(Y))=C^\prime$,很显然不能增大容量。

(b)由(a)知,仅在等号成立时不会严格减少容量。

此时$I(X;Y)= I(X;g(Y))$,当且仅当$I(X;Y|g(Y))=0$(即$X\rightarrow g(Y)\rightarrow Y$构成马尔科夫链)时等号成立。


\subsection*{7.2•可加噪声信道}
$Y = X + Z$

\begin{itemize}
    \item 当a=0时,$Y=X, C=max_{p(x)}I(X,Y)=max_{p(x)}H(X)=1 bit$
    \item 当a=1时,$\mathcal{Y} =\{0,1,2\}, C=max_{p(x)}I(X,Y)=max_{p(x)}H(X)-H(X|Y)=1-\frac{1}{2}=\frac{1}{2} bit$
    \item 当a=-1时,$\mathcal{Y} =\{-1,0,1\}, C=max_{p(x)}I(X,Y)=max_{p(x)}H(X)-H(X|Y)=\frac{1}{2} bit$
    \item 当$a\neq 0,\pm 1$时,$\mathcal{Y} =\{0,1,a,1+a\}, C=max_{p(x)}I(X,Y)=max_{p(x)}H(X)-H(X|Y)=1 bit$
\end{itemize}
\subsection*{7.5•同时使用两个信道}
\begin{equation*}
    \begin{split}
        C=&maxI(X_1,X_2;Y_1,Y_2)\\
        =&max[H(Y_1,Y_2)-H(Y_1,Y_2|X_1,X_2)]\\
        =&max[H(Y_1,Y_2)-H(Y_1|X_1,X_2)-H(Y_2|X_1,X_2)]\\
        =&max[H(Y_1,Y_2)-H(Y_1|X_1)-H(Y_2|X_2)]\\
        \leqslant & max[H(Y_1)+H(Y_2)-H(Y_1|X_1)-H(Y_2|X_2)]\\
        =& max[I(X_1;Y_1)+I(X_2;Y_2)]\\
        =& C_1+C_2
    \end{split}
\end{equation*}

当且仅当$H(Y_1,Y_2)= H(Y_1)+H(Y_2)$时取等号,此时有
\begin{equation*}
    \begin{split}
        p(y_1,y_2)&=p(y_1,y_2|x_1,x_2)p(x_1,x_2)\\
        &=p(y_1|x_1)p(y_2|x_2)p(x_1,x_2)\\
        &=p(y_1)p(y_2)\\
        &=p(y_1|x_1)p(x_1)p(y_2|x_2)p(x_2)
    \end{split}
\end{equation*}

得$p(x_1,x_2)=p(x_1)p(x_2)$,其中$x_1,x_2$分别取取得$C_1,C_2$ 时的分布。
\subsection*{7.13•二元信道中的擦除与出错}
(a)\begin{equation*}
    \begin{split}
        C &= max_{p(x)}I(X;Y)\\
        &= max_{p(x)}(H(Y)-H(Y|X))\\
        &= max_{p(x)}H(Y)-H(1-\epsilon-\alpha, \epsilon, \alpha)\\
    \end{split}
\end{equation*}
设$p(x=1)=\pi, p(x=0)=1-\pi$,则
\begin{equation*}
    \begin{split}
        H(Y) &= H((1-\pi)(1-\epsilon-\alpha)+\epsilon\pi, \alpha, \pi(1-\alpha-\epsilon)+(1-\pi)\epsilon)\\
        &= H(\alpha) + (1-\alpha)H(\dfrac{(1-\pi)(1-\epsilon-\alpha)+\epsilon\pi}{1-\alpha},\dfrac{\pi(1-\alpha-\epsilon)+(1-\pi)\epsilon}{1-\alpha})\\
        &\leqslant H(\alpha)+(1-\alpha)
    \end{split}
\end{equation*}
当且仅当$\dfrac{(1-\pi)(1-\epsilon-\alpha)+\epsilon\pi}{1-\alpha}=\dfrac{\pi(1-\alpha-\epsilon)+(1-\pi)\epsilon}{1-\alpha}=\frac{1}{2}$时取等号,此时$ \pi=\frac{1}{2}$.

\begin{equation*}
    \begin{split}
        \therefore C &= max_{p(x)}H(Y)-H(1-\epsilon-\alpha, \epsilon, \alpha)\\
        &= H(\alpha)+(1-\alpha)-H(1-\epsilon-\alpha, \epsilon, \alpha)\\
        &= (1-\alpha)(1-H(\dfrac{1-\epsilon-\alpha}{1-\alpha},\dfrac{\epsilon}{1-\alpha}))
    \end{split}
\end{equation*}

(b)$C=1-H(\epsilon)$

(c)$C=1-\alpha$

\subsection*{7.18•信道容量}
(a)$$C=max_{p(x)}I(X;Y)=log|\mathcal{Y} |-H(\textbf{r})=log3-log3=0 bit $$

(b)$$ C=max_{p(x)}I(X;Y)=log|\mathcal{Y} |-H(\textbf{r})=log3-1 \approx 0.58bit $$

(c)这个信道可以看做是两个信道并联,则有$C_1 = 1-H(p),C_2 = 1-H(q)$,由$2^C = 2^{C_1}+2^{C_2}$得
$$C = log(2^{1-H(p)}+2^{1-H(q)}) $$
% 设$p_i=Pr(x=i)$
% \begin{equation*}
%     \begin{split}
%         C=&max_{p(x)}I(X;Y)\\
%         =&max_{p(x)}(H(Y)-H(Y|X)) \\
%         =&(p_0p+p_1(1-p))log\frac{1}{p_0p+p_1(1-p)}+(p_0(1-p)+p_1p)\frac{1}{p_0(1-p)+p_1p}\\
%         &+(p_2+p_3(1-q))log\frac{1}{p_2q+p_3(1-q)}+(p_2(1-q)+p_3q)\frac{1}{p_2(1-q)+p_3q}\\
%         &-(p_0+p_1)H(p)-(p_2+p_3)H(q)\\
%         =& \Box bit
%     \end{split}
% \end{equation*}
% 这个真不会了,咋做啊……大胆猜测$p_0=p_1,p_2=p_3$后面就不会了
\subsection*{7.20•在输出Y上带两个独立观察的信道}
(a)\begin{equation*}
    \begin{split}
        I(X;Y_1,Y_2)=&H(Y_1,Y_2)-H(Y_1,Y_2|X)\\
        =& H(Y_1)+H(Y_2|Y_1)-H(Y_1,Y_2|X)\\
        \because Y_1,Y_2 iid\Rightarrow &I(Y_1;Y_2)=0, H(Y_1)=H(Y_2)\\
        \therefore I(X;Y_1,Y_2)=&H(Y_1)+H(Y_2)-H(Y_1|X)-H(Y_2|X)\\
        =& 2I(X;Y_1)\\
        =&2I(X;Y_1)-I(Y_1;Y_2)
    \end{split}
\end{equation*}

(b)

信道$X\rightarrow Y_1$的容量$C=maxI(X;Y_1)$;

信道$X\rightarrow (Y_1,Y_2)$容量$C^\prime = max I(X;Y_1,Y_2)=max (2I(X;Y_1)-I(Y_1;Y_2))$;

$\because I(Y_1;Y_2)\geqslant 0$

$\therefore C^\prime \leqslant 2C$

\subsection*{7.22•添加信号会降低容量吗?}
信道转移矩阵$p(y|x)$表示x行y列,添加一行表示添加了一个x,对应概率均为0,此时H(Y),H(Y|X)均不受影响,故容量不变。
\subsection*{7.28•信道的选取}
(a)设两个信道组成的新信道的输出为Z,$Pr(Z=Y_1)=\alpha,Pr(Z=Y_2)=(1-\alpha)$,由题意$X\rightarrow Y\rightarrow Z$,即$I(X;Z|Y)=0$,则有
\begin{equation*}
    \begin{split}
        C = &max I(X;Y,Z)\\
        = & max(I(X;Y|Z)+I(X;Z))\\
        = & max (I(X;Z|Y)+I(X;Y))\\
        = & maxI(X;Y)\\
        \therefore I(X;Y)=&I(X;Y|Z)+I(X;Z)\\
        =&\alpha I(X;Y_1)+(1-\alpha)I(X;Y_2)+H(Z)-H(Z|X)\\
        =&\alpha I(X;Y_1)+(1-\alpha)I(X;Y_2)+H(\alpha)\\
        \therefore C = &max (\alpha C_1 +(1-\alpha)C_2 +H(\alpha))
    \end{split}
\end{equation*}
对$\alpha$求导得,$C\prime = C_1-C_2 -log\alpha -1 +log (1-\alpha)+1=C_1-C_2-log\frac{\alpha}{1-\alpha}$,在$C_1-C_2=log\frac{\alpha}{1-\alpha}$时取得极大值$C = C_1+log(1+\frac{2^{C_1}}{2^{c_2}}$,此时$2^C=2^{C_1}\times (1+\frac{2^{C_1}}{2^{c_2}})=2^{C_1}+2^{C_2}$

(b)虽然这个b不用做,但是我2-10的那个解释就不太会,能不能给个标准答案啊,非常感谢!

(c)$C_1 = 1-H(p)$

$C_2 = 0$

$C=log(2^{C_1}+2^{C_2})=log(2^{1-H(p)}+1)$
\subsection*{7.29•信源与信道}
$C = 1-H(p)$

$H (\mathcal{V} )=H(\alpha)$

$\therefore H(\alpha)< 1-H(p)$

\subsection*{7.32•信道容量}
(a)$C = log(2\times 2^{1-H(p)})=2-H(p)$

(b)$C = log(2^{C_1}+2^{C_2})=log(2^{1-H(p)}+1)$

(c)$C_1 = log3 - 1 =log\frac{3}{2}bit$

\quad \ $C_2 = log2 - 1 = 0bit$

\quad $\therefore C = log(2^{C_1}+2^{C_2})=log\frac{5}{2}$

(d)弱对称$C = log|\mathcal{Y}|-H(\frac{2}{3},\frac{1}{3})= log3 +\frac{2}{3}log\frac{2}{3}+\frac{1}{3}log\frac{1}{3}=\frac{2}{3}bit$

\subsection*{7.33•信道容量}
(a)相当于$\mathcal{P}$与一个单字符信道并联;
$C_1 = C , C_2 = 0 \Rightarrow\tilde{C} = log(1+2^C) $

(b)相当于$\mathcal{P}$与一个传输矩阵为$I_k$的信道并联;$C_1 = C , C_2 = logk \Rightarrow\tilde{C} = log(k+2^C) $

\end{document}
