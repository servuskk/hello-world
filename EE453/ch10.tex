%用来打印中文
\documentclass[UTF8]{ctexart}
\RequirePackage{iftex}
\RequirePackage{fix-cm}
\RequirePackage{fixltx2e}
%页面设置
\RequirePackage{geometry}
\if@twoside
 \geometry{a4paper,
 bindingoffset = 2cm,
 inner = 0.5cm,
 outer = 2cm,
 top = 3cm,
 bottom = 2cm
 }
\else
 \geometry{a4paper,
 left = 2cm,
 right = 2cm,
 top = 2cm,
 bottom = 2cm
 }
%各种包
\usepackage{fancyhdr}
\usepackage{amssymb}
\RequirePackage{graphicx}
\RequirePackage{subfigure}
\RequirePackage{caption}
\RequirePackage{diagbox}
\RequirePackage{multirow}
\RequirePackage{makecell}
\RequirePackage{booktabs}
%\usepackage{lipsum}
\usepackage{mathtools}
\usepackage{listings}%代码
%矩阵
\usepackage{amsmath,xcolor}
\RequirePackage{longtable}
\RequirePackage{array}
%页眉
\RequirePackage{float}
\RequirePackage{flowchart}
\pagestyle{fancy}
\renewcommand{\headrulewidth}{0.5pt} 
\lhead{} \rhead{whh\ allesgutewh@gmail.com}%\thepage
% 顶格写\noindent
\begin{document}
%\pagenumbering{arabic}
\section*{Information Theory Homework-ch10}
%姓名(学号)第x章作业.pdf
\begin{center}
\today
\end{center}
\subsection*{Assignments}


10.1,\ 10.5,\  10.8,\  10.14 


\subsection*{10.1•单个高斯随机变量的1bit量化}
设最佳再生点为$a_1,a_2$,由题意$X\sim\mathcal{N}(0,\sigma^2)$,有$a_1=-a_2=a$,不妨设a>0则
\begin{equation*}
    \begin{split}
        a = &\frac{\int_0^\infty xf(x)dx}{\int_0^\infty f(x)dx}\\
        =&\int_{0}^{\infty}\frac{2x}{\sqrt{2\pi\sigma^2}}e^{-\frac{x^2}{2\sigma^2}}dx\\
        =&\sqrt{\frac{\sigma^2}{2\pi}}\int_0^\infty e^{-t}dt\\
        =& \sqrt{\frac{2}{\pi}}\sigma
    \end{split}
\end{equation*}
期望失真:
\begin{equation*}
    \begin{split}
        2\int_0^\infty f(x)(x-a)^2dx=& \sigma^2 -4a\int_0^\infty xf(x)dx +a^2\\
        =&\sigma^2 - a^2\\
        =& (1-\frac{2}{\pi})\sigma^2
    \end{split}
\end{equation*}

$D = \sigma^22^{-2R}=\frac{\sigma^2}{4}$

$\frac{1}{4}=0.25$

$1-\frac{2}{\pi}\approx 0.36 > 0.25$

$\therefore (1-\frac{2}{\pi})\sigma^2>D$
\subsection*{10.5•具有汉明失真度量的均匀分布信源的率失真}
\begin{equation*}
    \begin{split}
        R(D)=& min I(X;\hat{X})\\
        I(X;\hat{X})=& H(X) - H(X|\hat{X})\\
        \geqslant &  log m - H(D) - D\log(m-1)\ (P22\ Fano)\\
    \end{split}
\end{equation*}
$1-D = \frac{D}{m-1}\Rightarrow D = \frac{m-1}{m}$时等号成立。

$\therefore R(D)=(log m - H(D) - D\log(m-1))^+\ (D>\frac{m-1}{m}$时取0$)$
\subsection*{10.8•平方误差失真度量意义下的率失真函数的界}
\begin{equation*}
    \begin{split}
        R(D)=& min I(X;\hat{X})\\
        I(X;\hat{X})=& h(X) - h(X|\hat{X})\\
            =& h(X) - h(X-\hat{X}|\hat{X})\\
            \geqslant & h(X) - h(X-\hat{X})\\
            \geqslant & h(X) - \frac{1}{2}\log 2\pi e D\\
    \end{split}
\end{equation*}

$\therefore R(D)\geqslant h(X) - \frac{1}{2}\log 2\pi e D$

图中信道$\hat{X}=\frac{\sigma^2-D}{\sigma^2}(X+Z)$

$E(X-\hat{X})^2=\frac{1}{\sigma^4}E(DX - (\sigma^2 - D)Z)^2 = \frac{1}{\sigma^4}(D^2\sigma^2 +D\sigma^4 -D^2\sigma^2)=D$。

$I(X;\hat{X})=h(\hat{X})-h(\frac{\sigma^2-D}{\sigma^2}Z)=\frac{1}{2}\log (\sigma^2-D)-\frac{1}{2}\log (\frac{D(\sigma^2-D)}{\sigma^2})=\frac{1}{2}\log \frac{\sigma^2}{D}$

$\because R(D)=min I(X;\hat{X})$

$\therefore R(D)\leqslant \frac{1}{2}\log \frac{\sigma^2}{D}$

$\therefore h(X) - \frac{1}{2}\log 2\pi e D\leqslant R(D)\leqslant \frac{1}{2}\log \frac{\sigma^2}{D}$


描述起来更容易,因为高斯随机变量的$h(X)=\frac{1}{2}log2\pi e\sigma^2$,上述不等式的上界下界相同,不需要用不等式描述。
\subsection*{10.14•两个独立信源的率失真}
(a)由$X_i, Y_i$独立知,$h(XY)=h(X)+h(Y)$
\begin{equation*}
    \begin{split}
        R_{X,Y}(D)=& min I(X,Y;\hat{X},\hat{Y})\\
        I(X,Y;\hat{X},\hat{Y})=& h(XY) - h(XY|\hat{X}\hat{Y})\\
            =& h(X)+h(Y) - h(X|\hat{X}\hat{Y})-h(Y|X\hat{X}\hat{Y})\\
            \geqslant & h(X) - h(X|\hat{X})+h(Y) - h(Y|\hat{Y})\\
            =& I(X;\hat{X})+I(Y;\hat{Y})\\
    \end{split}
\end{equation*}
$\therefore R_{X,Y}(D)=min_{X,Y} I(X,Y;\hat{X},\hat{Y})\geqslant min_X I(X;\hat{X})+ min_YI(Y;\hat{Y})=R_Y(D)+R_X(D)$

(b)$I(X,Y;\hat{X},\hat{Y})\geqslant I(X;\hat{X})+I(Y;\hat{Y})$等号成立的条件是$h(X|\hat{X}\hat{Y}) =  h(X|\hat{X});\ h(Y|X\hat{X}\hat{Y})=h(Y|\hat{Y})$。该条件在X Y、$\hat{X} \hat{Y}$相互独立时是成立的,不妨假设当满足独立条件时有分布取得$R_X(D_1)=min_{p(\hat{x}|x):Ed(x-\hat{x}\leqslant D_1)}I(X;\hat{X})$、$R_Y(D_2)=min_{p(\hat{y}|y):Ed(x-\hat{y}\leqslant D_2)}I(Y;\hat{Y})$,这两个分布相加即可得到$R_{X,Y}(D)$,等式成立。

在XY及$\hat{X}\hat{Y}$独立时,没有区别。
\end{document}
