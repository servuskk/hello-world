%用来打印中文
\documentclass[UTF8]{ctexart}
\RequirePackage{iftex}
\RequirePackage{fix-cm}
\RequirePackage{fixltx2e}
%页面设置
\RequirePackage{geometry}
\if@twoside
 \geometry{a4paper,
 bindingoffset = 2cm,
 inner = 0.5cm,
 outer = 2cm,
 top = 3cm,
 bottom = 2cm
 }
\else
 \geometry{a4paper,
 left = 2cm,
 right = 2cm,
 top = 2cm,
 bottom = 2cm
 }
%各种包
\usepackage{fancyhdr}
\usepackage{amssymb}
\RequirePackage{graphicx}
\RequirePackage{subfigure}
\RequirePackage{caption}
\RequirePackage{diagbox}
\RequirePackage{multirow}
\RequirePackage{makecell}
\RequirePackage{booktabs}
%\usepackage{lipsum}
\usepackage{mathtools}
\usepackage{listings}%代码
%矩阵
\usepackage{amsmath,xcolor}
\RequirePackage{longtable}
\RequirePackage{array}
%页眉
\RequirePackage{float}
\RequirePackage{flowchart}
\pagestyle{fancy}
\renewcommand{\headrulewidth}{0.5pt} 
\lhead{} \rhead{whh\ allesgutewh@gmail.com}%\thepage
% 顶格写\noindent
\begin{document}
%\pagenumbering{arabic}
\section*{Information Theory Homework-ch8}
%姓名(学号)第x章作业.pdf
\begin{center}
\today
\end{center}
\subsection*{Assignments}


8.1\ 8.2\ 8.5\ 8.8\ 8.9


\subsection*{8.1•微分熵}
(a)\begin{equation*}
    \begin{split}
        h(X)=&-\int f(x)\ln f(x) \,dx \\
            =&-\int_0^\infty \lambda e^{-\lambda x}(\ln\lambda - \lambda x)\,dx \\
            =&\ln\lambda e^{-\lambda x }\vert _0^\infty-\lambda (x+\frac{1}{\lambda}) e^{-\lambda x}\vert _0^\infty\\
            =& -\ln\lambda +1 \\
            =& \ln \dfrac{e}{\lambda}\\
            =&\log \dfrac{e}{\lambda} bit
    \end{split}
\end{equation*}


(b)\begin{equation*}
    \begin{split}
        h(X)=&-\int f(x)\ln f(x) \,dx \\
            =& \frac{1}{2}\ln \dfrac{e}{\lambda} -\frac{1}{2}\int_{-\infty}^0\lambda e^{\lambda x}(\ln\lambda + \lambda x)\,dx \\
            =& \frac{1}{2}\ln \dfrac{e}{\lambda} -\frac{1}{2}[\ln\lambda -1]\\
            =& \ln \dfrac{e}{\lambda}\\
            =&\log \dfrac{e}{\lambda} bit
    \end{split}
\end{equation*}


(c)由题意,$X_1+X_2\sim \mathcal{N}(\mu_1+\mu_2, \sigma^2_1+\sigma^2_2) $

\begin{equation*}
    \begin{split}
        h(X)=&-\int f(x)\ln f(x) \,dx \\
            =&\frac{1}{2}\log 2\pi e (\sigma_1^2+\sigma_2^2)
    \end{split}
\end{equation*}

\subsection*{8.2•行列式的凹性}

假设$X_1\sim N(0,K_1), X_2\sim N(0,K_2), \theta = Bernoulli(\lambda), Z=X_\theta\Rightarrow K_Z=\lambda K_1+\bar\lambda K_2$,(注意不一定是高斯分布)

$$\therefore h(Z|\theta)=p(\lambda)h(X_1)+(1-p(\lambda))h(X_2)=\lambda \frac{1}{2}log(2\pi e)^n |K_1|+\bar{\lambda}\frac{1}{2}log(2\pi e)^n|K_2|=\frac{1}{2}log(2\pi e)^n+\frac{1}{2}log |K_1|^\lambda  |K_2|^{\bar{\lambda}}$$

$$h(Z)\leqslant h(X_\theta)=\frac{1}{2}log(2\pi e)^n + \frac{1}{2}|\lambda K_1+\bar{\lambda}K_2|$$

$$\because h(Z|\theta)\leqslant h(Z)$$

$$\therefore |K_1|^\lambda  |K_2|^{\bar{\lambda}}\leqslant |\lambda K_1+\bar{\lambda}K_2|$$
\subsection*{8.5•尺度性质。设$h(X)=-\int f(x)logf(x)dx$,证明$h(AX)=log|det(A)|+h(X)$。}

令$Y=AX$。则$f_Y(y)=\frac{1}{|det(A)|}f_X(A^{-1}y)$

\begin{equation*}
    \begin{split}
        h(AX)=&-\int f_Y(y)log (f_Y(y))\,dy\\
            =&-\int \frac{1}{|det(A)|}f_X(A^{-1}y)log(\frac{1}{|det(A)|}f_X(A^{-1}y))dy\\
            =& log |det(A)| + h(X)
    \end{split}
\end{equation*}

\subsection*{8.8•有均匀干扰噪声的信道}
Consider a additive channel whose input alphabet $X = \{0,\pm 1, \pm 2\}$ , and whose output Y = X + Z , where Z is uniformly distributed over the interval [−1, 1] . Thus the input of the channel is a discrete random variable, while the output is continuous. Calculate the capacity $C = max_{p(x)} I(X; Y )$
of this channel.

提示:首先证明,若限定连续型随机变量X的取值范围在[a,b]上时,均匀分布使得X的微分熵最大。

证明:$X\in [a, b]$,定义一个随机变量$X^{\Delta}\in \{x_i,\cdots ,x_i,\cdots\}, p(x_i)=f(x_i)\cdot\Delta$,则$X^{\Delta}$的熵为
\begin{equation*}
    \begin{split}
        H(X^\Delta)=&-\sum _ip(x_i)log p(x_i)\\
        =& -\sum _if(x_i)\Delta log f(x_i)\Delta\\
        =&-\sum _i[f(x_i)\Delta ] log f(x_i) -log \Delta\\
    \end{split}
\end{equation*}
如果$f(x)logf(x)$黎曼可积,则当$\Delta \rightarrow 0$时,$H(X^\Delta)+log\Delta \rightarrow h(X)$, 由$H(X^\Delta)$在$p(x_i)=p(x_j)=\frac{1}{N}=\frac{\Delta}{b-a}$时取最大值可知,均匀分布使得X的微分熵最大。

在本题中,$Z\in[-3, 3]$,由题意知,当X的取值分布为$p(X=-2)=p(X=0)=p(X=2)=\frac{1}{3}$时,Y在[-3,3]内均匀分布,取得$h(Y)_{max}=log6$

\begin{equation*}
    \begin{split}
        C & = max_{p(x)} I(X; Y)\\
        & = max_{p(x)}[h(Y)-h(Y|X)]\\
        & = max_{p(x)}[h(Y)-h(Z)]\\
        & = max_{p(x)}H(Y) - log2\\
        & = log6 - log2\\
        & = log3
    \end{split}
\end{equation*}
\subsection*{8.9•高斯互信息。假设(X,Y,Z)是联合高斯分布,并且$X\rightarrow Y\rightarrow Z$构成一个马尔可夫链。令(X,Y)与(X,Z)的相关系数分别为$\rho _1, \rho _2$。求I(X;Z)}
由题意,X与Z的协方差矩阵
\[
    K = \begin{bmatrix}
     \sigma_x^2 & \sigma_x\sigma_z\rho_{xz} \\
     \sigma_x\sigma_z\rho_{xz}& \sigma_y^2
    \end{bmatrix}
\]

\begin{equation*}
     \begin{split}
        \therefore I(X;Z)=& h(X)+h(Z)-h(X,Z)\\
        =& \frac{1}{2}log 2\pi e\sigma_x^2+\frac{1}{2}log 2\pi e\sigma_z^2-\frac{1}{2}log (2\pi e)^2|K|\\
        =&\frac{1}{2}log \frac{\sigma_x^2\sigma_y^2}{\sigma_x^2\sigma_y^2(1-\rho_{xz}^2)}\\
        =&-\frac{1}{2}log (1-\rho_{xz}^2)
    \end{split}
\end{equation*}

不妨假设X,Y,Z的均值均为0,则有
\begin{equation*}
    \begin{split}
        \rho_{xz}=&\dfrac{E[XZ]}{\sigma_x \sigma_z}\\
        =& \dfrac{E(E[XZ|Y])}{\sigma_x \sigma_z}\\
        =& \dfrac{E(E[X|Y]E[Z|Y])}{\sigma_x \sigma_z}\\
        =& \dfrac{\sigma _x\rho_1\sigma_z\rho_2}{\sigma_x \sigma_z}\\
        =& \rho_1\rho_2
    \end{split}
\end{equation*}
$$\therefore I(X;Z)=-\frac{1}{2}\log (1-\rho_1^2\rho_2^2)$$
\end{document}
