%用来打印中文
\documentclass[UTF8]{ctexart}
\RequirePackage{iftex}
\RequirePackage{fix-cm}
\RequirePackage{fixltx2e}
%页面设置
\RequirePackage{geometry}
\if@twoside
 \geometry{a4paper,
 bindingoffset = 2cm,
 inner = 0.5cm,
 outer = 2cm,
 top = 3cm,
 bottom = 2cm
 }
\else
 \geometry{a4paper,
 left = 2cm,
 right = 2cm,
 top = 2cm,
 bottom = 2cm
 }
%各种包
\usepackage{fancyhdr}
\usepackage{amssymb}
\RequirePackage{graphicx}
\RequirePackage{subfigure}
\RequirePackage{caption}
\RequirePackage{diagbox}
\RequirePackage{multirow}
\RequirePackage{makecell}
\RequirePackage{booktabs}
%\usepackage{lipsum}
\usepackage{mathtools}
\usepackage{listings}%代码
%矩阵
\usepackage{amsmath,xcolor}
\RequirePackage{longtable}
\RequirePackage{array}
%页眉
\RequirePackage{float}
\RequirePackage{flowchart}
\pagestyle{fancy}
\renewcommand{\headrulewidth}{0.5pt} 
\lhead{} \rhead{name\ number}%\thepage
% 顶格写\noindent
\begin{document}
%\pagenumbering{arabic}
\section*{Discrete Mathematics Homework-6}
\begin{center}
\today
\end{center}
\subsection*{Assignment}

(P155)\ 7(5)\quad 8\quad 12 \quad 17(1,\ 4)\quad 27

\subsection*{•7写出集合$P(P(\emptyset))\times P(P(\emptyset)) $}
% 集合的幂集是该集合所有自己组成的集合。幂集是有一个集合构造的新集合,他也是集合的一元运算。但是幂集与原集合的层次有所不同。

$ P(\emptyset)=\{\emptyset \}$

$ P(P(\emptyset))=P(\{\emptyset \})=\{\emptyset , \{\emptyset \}\}$


% 笛卡儿积也是一种集合二元运算。两个集合的笛卡儿积是他们的元素组成的有序对的集合。笛卡儿积是与原集合层次不用的集合。

% 有序对$<x,y> $定义为$$ <x,y>=\{\{x\},\{x,y\}\}$$

% 集合A,B的笛卡儿积$A\times B$定义为$$A\times B=\{<x,y>\mid x\in A\wedge y\in B \} $$

$P(P(\emptyset))\times P(P(\emptyset))=\{\emptyset , \{\emptyset \}\}\times \{\emptyset , \{\emptyset \}\}=\{ <\emptyset,\emptyset> ,<\emptyset, \{\emptyset \}>,< \{\emptyset \},\emptyset>,< \{\emptyset \}, \{\emptyset \}>\}$
\subsection*{•8设$B=P(P(P(\emptyset)))$}

$ P(\emptyset)=\{\emptyset \}$

$ P(P(\emptyset))=P(\{\emptyset \})=\{\emptyset , \{\emptyset \}\}$

$ P(P(P(\emptyset)))=\{\emptyset ,\{\emptyset \},\{\{\emptyset \} \}, \{\emptyset , \{\emptyset \}\} \}$
\subsubsection*{(1)是否$\emptyset \in B$?是否$\emptyset \subseteq B$?}

$\emptyset \in B$

$\emptyset \subseteq B$
\subsubsection*{(2)是否$\{\emptyset\}\in B $?是否$\{\emptyset\}\subseteq B $?}

$\{\emptyset\}\in B $

$\{\emptyset\}\subseteq B $
\subsubsection*{(3)是否$\{\{\emptyset\}\}\in B $ ?是否$\{\{\emptyset\}\}\subseteq B $?}

$\{\{\emptyset\}\}\in B $

$\{\{\emptyset\}\}\subseteq B $
\subsection*{•12设全集$E=\{1,2,3,4,5 \} $,集合$A=\{1,4\} $,$B=\{1,2,5 \} $,$C=\{2,4\}$.求下列集合:}
\subsubsection*{(1)$A\bigcap -B =\{4 \}$}

\subsubsection*{(2)$(A\bigcap B) \cup -C=\{1,3,5 \}$}
\subsubsection*{(3)$-(A\bigcap B) =\{2,3,4,5 \}$}
\subsubsection*{(4)$P(A)\bigcap P(B) =\{\emptyset ,\{1\} \}$}
\subsubsection*{(5)$P(A)-P(B) =\{\{4\},\{1,4\} \}$}

\subsection*{•17设A,B和C是任意集合,证明:}
\subsubsection*{(1)$(A-B)-C=A-(B\cup C) $}

$A-B=A\bigcap -B$

$(A-B)-C=A\bigcap -B\bigcap -C=A\bigcap -(B\cup C)=A-(B\cup C)$

\subsubsection*{(4)$A\subseteq C\wedge B\subseteq C\Leftrightarrow A\cup B\subseteq C $}

$A\subseteq C\wedge B\subseteq C\Rightarrow (A\cup B)\subseteq (C\cup C)\Rightarrow A\cup B\subseteq C$

$A\cup B\subseteq C\Rightarrow ((A\cup B)\bigcap A \subseteq (C\bigcap A))\wedge ((A\cup B)\bigcap B \subseteq (C\bigcap B))$


\quad \quad \quad \quad \quad $\Rightarrow (A\subseteq ( C\bigcap A) )\wedge (B\subseteq ( C\bigcap B)$


\quad \quad \quad \quad \quad $\Rightarrow A\subseteq C\wedge B\subseteq C$

得证。
\subsection*{•27足球队有38人,篮球队有15人,排球队有20人,三个队队员共58人,其中3人同时参加三个队,问同时参加两个队的有几个。}

$58=38+15+20-x_1-x_2-x_3+3$

$x=(x_1-3)+(x_2-3)+(x_3-3)=12$\ (不含同时参加三个)

\end{document}
