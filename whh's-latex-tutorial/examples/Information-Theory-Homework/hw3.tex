%用来打印中文
\documentclass[UTF8]{ctexart}
\RequirePackage{iftex}
\RequirePackage{fix-cm}
\RequirePackage{fixltx2e}
%页面设置
\RequirePackage{geometry}
\if@twoside
 \geometry{a4paper,
 bindingoffset = 2cm,
 inner = 0.5cm,
 outer = 2cm,
 top = 3cm,
 bottom = 2cm
 }
\else
 \geometry{a4paper,
 left = 2cm,
 right = 2cm,
 top = 2cm,
 bottom = 2cm
 }
%各种包
\usepackage{fancyhdr}
\usepackage{amssymb}
\RequirePackage{graphicx}
\RequirePackage{subfigure}
\RequirePackage{caption}
\RequirePackage{diagbox}
\RequirePackage{multirow}
\RequirePackage{makecell}
\RequirePackage{booktabs}
%\usepackage{lipsum}
\usepackage{mathtools}
\usepackage{listings}%代码
%矩阵
\usepackage{amsmath,xcolor}
\RequirePackage{longtable}
\RequirePackage{array}
%页眉
\RequirePackage{float}
\RequirePackage{flowchart}
\pagestyle{fancy}
\renewcommand{\headrulewidth}{0.5pt} 
\lhead{} \rhead{姓名(学号)第五章作业}%\thepage
% 顶格写\noindent
\begin{document}
%\pagenumbering{arabic}
\section*{Information Theory Homework-3}
%姓名(学号)第五章作业.pdf
\begin{center}
\today
\end{center}
\subsection*{Assignments}

5.2,\ 5.6,\ 5.21,\ 5.28,\ 5.32,\ 5.33,\ 5.37,\ 5.41.
\subsection*{5.2•火星人有多少个手指头?}
由Kraft不等式,对于D元字母表有$$\sum_i D^{-l_i}\leqslant 1 $$

$\because l_1=l_2=1\Rightarrow 2^{-l_1}+2^{-l_2}=1$

$\therefore D>2$

$(3^{-1}+3^{-2}+3^{-3})\times 2=\frac{26}{27}<1 $

$\therefore$下界为3

三进制,三个手指头。

\subsection*{5.6•坏码}

(a)可以是赫夫曼码

(b)可以使用更短的$\{00,01,10,11\}$,不可能称为霍夫曼码。

(c)可以使用更短的$\{0,1\}$,不可能称为霍夫曼码。

\subsection*{5.21•惟一可译性的成立条件}
%\noindent
码C是惟一可译$\Leftrightarrow$对任意$k\geqslant 1$,展开式$C^k(x_1,x_2,\cdots ,x_k)=C(x_1)C(x_2)\cdots C(x_k)$是$\chi ^k$到$D^k$的1-1映射

后者是前者的必要条件(左到右)是显然的,证明充分性(右到左):

由扩展码是$\chi ^k$到$D^k$的1-1映射,得扩展码是非奇异的,由UD码定义“如果一个编码的扩展编码是非奇异的,则称该编码是唯一可译的”得码C是唯一可译的。

%若码C不是唯一可译的,则有$C^k(x_1,x_2,\cdots ,x_k)=C(x_1)C(x_2)\cdots C(x_k)=C(y_1)C(y_2)\cdots C(y_k)$,其中$(x_i-y_i)$不全为零,则$C^k(y_1,y_2,\cdots ,y_k)=C^k(x_1,x_2,\cdots ,x_k)$不是$\chi ^k$到$D^k$的1-1映射,故后者是前者的充分条件。

\subsection*{5.28•香农码}
(a)
由$l_i=\lceil log\frac{1}{p_i}\rceil$,得$log\frac{1}{p_i}\leqslant l_i< log\frac{1}{p_i}+1\Rightarrow H(X)\leqslant L<H(X)+1$

由$l_i=\lceil log\frac{1}{p_i}\rceil$,得$2^{-l_i}\leqslant p_i<2^{-(l_i-1)}$,则由F定义当$j>i$时,$F_{j}-F_i\geqslant 2^{-l_i}$,由于$F_i$对应码字有$l_i$位,由$p_1\geqslant p_2\geqslant \cdots \geqslant p_m$,得$l_j=\lceil log\frac{1}{p_j}\rceil \geqslant \lceil log\frac{1}{p_i}\rceil= l_i$,即$F_j$对应的码字长度不小于$l_i$,结合$F_j-F_i\geqslant 2^{-l_i}$有$F_j$的前$l_i$位与$F_i$必不相同,故该编码是无前缀的。

(b)编码如表1所示。
\begin{table}[h]
\begin{center}
\caption{5.28-(b)编码}
\begin{tabular}{ccccc}
\hline
$i$ & $p_i$ & $l_i$ & $F_i$ & $C_i$\\
\hline
1 & $\frac{1}{2}$ & 1 &$0$ & 0\\
2 & $\frac{1}{4}$ & 2 &$\frac{1}{2}$ & 10\\
3 & $\frac{1}{8}$ & 3 &$\frac{3}{4}$ & 110\\
4 & $\frac{1}{8}$ & 3 &$\frac{7}{8}$ & 111\\
\hline
\end{tabular}
\end{center}
\end{table}


\subsection*{5.32•坏葡萄酒}
(a)不妨设要品尝次数为X,则
$E(X)=1\times \frac{8}{23}+2\times \frac{6}{23}+3\times\frac{4}{23}+4\times\frac{2}{23}+5\times (\frac{2}{23}+\frac{1}{23}=\frac{55}{23}\approx 2.39$次


(b)$p_i$最大的一瓶

(a)$E(X)=2\times (\frac{8}{23}+\frac{6}{23}+\frac{4}{23})+3\times\frac{2}{23}+4\times (\frac{2}{23}+\frac{1}{23})=\frac{54}{23}\approx 2.35$次

(b)$p_i$最大和次大的混合
\subsection*{5.33•赫夫曼与香农}
(a)

赫夫曼:$1\times 0.6+2\times(0.3+0.1)=1.4$

香农:$0.6\times \lceil log\frac{1}{0.6}\rceil+0.3\times \lceil log\frac{1}{0.3}\rceil+0.1\times \lceil log\frac{1}{0.1}\rceil=0.6\times 1+0.3\times 2+0.1\times 4=1.6  $

(b)由(a)知D=2时,二者不相等。X有三个值,则$D\geqslant 3$时,霍夫曼码需要1位;

香农码$0.6\times \lceil log_D\frac{1}{0.6}\rceil+0.3\times \lceil log_D\frac{1}{0.3}\rceil+0.1\times \lceil log_D\frac{1}{0.1}\rceil=1$时,有$\lceil log_D\frac{1}{0.6}\rceil =\lceil log_D\frac{1}{0.3}\rceil=\lceil log_D\frac{1}{0.1}\rceil$,由$0<\lceil log_D\frac{1}{0.6}\rceil\leqslant \lceil log_D\frac{1}{0.3}\rceil\leqslant \lceil log_D\frac{1}{0.1}\rceil$,有$\lceil log_D\frac{1}{0.1}\rceil\leqslant 1$,解得$D\geqslant 10$,故最小值为10.
\subsection*{5.37•码}
(a)唯一可译码的任一编码字符串只来源于唯一可能的信源字符串;

(b)即时码一定是唯一可译的且没有码字是其他码字的前缀。

$C_1:C_1(x_1)=C_1(x_3x_3)\Rightarrow$都不是;

$C_2:2^{-2}\times 3+ 2^{-3}\times 2 = 1$,且没有码字是其他码字的前缀$\Rightarrow$唯一可译且即时;

$C_3:2^{-1}+2^{-2}+\cdots < \dfrac{\frac{1}{2}}{1-\frac{1}{2}}=1$,且没有码字是其他码字的前缀$\Rightarrow$唯一可译且即时;

$C_4:C_4(x_1x_1)=C_4(x_2)\Rightarrow$都不是。

\subsection*{5.41•最优码}
$\alpha = 0 $时,$\tilde{p}_{10}=0$,\ $\tilde{l}_1=l_,\tilde{l}_2=l_2,\cdots , \tilde{l}_9=l_9,\tilde{l}_{10}=0,\tilde{l}_{11}=l_{10}$

$\alpha = 1 $时,$\tilde{p}_{11}=0$,\ $\tilde{l}_1=l_,\tilde{l}_2=l_2,\cdots , \tilde{l}_9=l_9,\tilde{l}_{10}=l_{10},\tilde{l}_{11}=0$

$0<\alpha <1 $时,$\tilde{l}_1=l_,\tilde{l}_2=l_2,\cdots , \tilde{l}_9=l_9,\tilde{l}_{10}=l_{10}+1,\tilde{l}_{11}=l_{10}+1$

\end{document}
