%用来打印中文
\documentclass[UTF8]{ctexart}
\RequirePackage{iftex}
\RequirePackage{fix-cm}
\RequirePackage{fixltx2e}
%页面设置
\RequirePackage{geometry}
\if@twoside
 \geometry{a4paper,
 bindingoffset = 2cm,
 inner = 0.5cm,
 outer = 2cm,
 top = 3cm,
 bottom = 2cm
 }
\else
 \geometry{a4paper,
 left = 2cm,
 right = 2cm,
 top = 2cm,
 bottom = 2cm
 }
%各种包
\usepackage{fancyhdr}
\usepackage{amssymb}
\RequirePackage{graphicx}
\RequirePackage{subfigure}
\RequirePackage{caption}
\RequirePackage{diagbox}
\RequirePackage{multirow}
\RequirePackage{makecell}
\RequirePackage{booktabs}
%\usepackage{lipsum}
\usepackage{mathtools}
\usepackage{listings}%代码
%矩阵
\usepackage{amsmath,xcolor}
\RequirePackage{longtable}
\RequirePackage{array}
%页眉
\RequirePackage{float}
\RequirePackage{flowchart}
\pagestyle{fancy}
\renewcommand{\headrulewidth}{0.5pt} 
\lhead{} \rhead{姓名(学号)第三章作业}%\thepage
% 顶格写\noindent
\begin{document}
%\pagenumbering{arabic}
\section*{Information Theory Homework-2}
%姓名(学号)第三章作业.pdf
\begin{center}
\today
\end{center}
\subsection*{Assignments}

3.1,\ 3.2,\ 3.4,\ 3.11.

\subsection*{3.1•马尔可夫不等式与切比雪夫不等式}

(a)马尔可夫不等式

设$X_1=\{x|x\in X,x\geqslant t\},P_1=Pr\{X\geqslant t\},X_2=\{x|x\in X,x<t\},P_2=Pr\{X< t\}=1-P_1$,则有$EX_1\geqslant t, 0\leqslant EX_2<t$, $$\therefore \dfrac{EX}{t}=\dfrac{1}{t}P_1EX_1+\dfrac{1}{t}P_2EX_2\geqslant \dfrac{EX_1}{t}P_1\geqslant P_1$$

例:$X=\{0,1\},P(X=0)=P(X=1)=\frac{1}{2},EX=\frac{1}{2},$则t=1时,有$Pr\{X\geqslant 1\}=\frac{1}{2},\frac{EX}{1}=\frac{1}{2}$使等号成立。


(b)切比雪夫不等式

由题意$EX=\sigma^2,\ Pr\{|Y-\mu|>\epsilon\}=Pr\{X>\epsilon^2\},\ \dfrac{\sigma^2}{\epsilon^2}=\dfrac{EX}{\epsilon^2}$,由(a)得$Pr\{X>\epsilon^2\}\leqslant \dfrac{EX}{\epsilon^2}$,则原不等式成立。

(c)弱大数定理

由题意$E\bar{Z_n}=\mu$,
\begin{equation*}
D\bar{Z}_n=E(\bar{Z}_n-\mu)^2=E[\frac{1}{n^2}(\sum_{i=1}^n (Z_i-\mu))^2]=\frac{1}{n^2}E[\sum_{i=1}^n(Z_i-\mu)^2]+\frac{2}{n^2}E[\sum_{i\neq j}(Z_i-\mu)(Z_j-\mu)]=\frac{n\sigma^2}{n^2}=\frac{\sigma^2}{n}
\end{equation*} 

设$X=(\bar{Z}_n-\mu)^2$,则$EX=\dfrac{\sigma^2}{n}$,由(a),(b)得不等式成立。

\subsection*{3.2•AEP与互信息}

\begin{equation*}
\begin{split}
\frac{1}{n}log\frac{p(X^n)p(Y^n)}{p(X^n,Y^n)}&=\frac{1}{n}log\prod\limits_{i=1}^n\frac{p(X_i)p(Y_i)}{p(X_i,Y_i)}\\
&=\frac{1}{n}\sum\limits_{i=1}^nlogp(X_i)+\frac{1}{n}\sum\limits_{i=1}^nlogp(Y_i)-\frac{1}{n}\sum\limits_{i=1}^nlogp(X_i,Y_i)\\
\end{split}
\end{equation*}

$$\therefore\lim\limits_{n\to+\infty}\frac{1}{n}log\frac{p(X^n)p(Y^n)}{p(X^n,Y^n)}=-H(X)-H(Y)+H(X,Y)=-I(X,Y)$$

\subsection*{3.4•AEP}

(a)$\surd$

(b)$\surd$

(c)由定理3.1.2,$|A^n|\leqslant 2^{n(H+\epsilon)}$,由$|A^n\cap B^n|\leqslant |A^n|$,有$$|A^n\cap B^n|\leqslant |A^n|\leqslant 2^{n(H+\epsilon)}$$

(d)设$\epsilon _0=\frac{1}{2}$,由AEP和大数定理有$P_r\{X^n\in A^n\cap B^n\}\to 1$,则存在N使得当$n>N$时有$P_r\{X^n\in A^n\cap B^n\}> 1-\epsilon _0$.同定理3.1.2的证明:

\begin{equation*}
\begin{split}
1-\epsilon _0&<P_r\{X^n\in A^n\cap B^n\}\\
&\leqslant \sum _{x^n\in A^n\cap B^n}2^{-n(H-\epsilon )}\\
&= 2^{-n(H-\epsilon )}|A^n\cap B^n|
\end{split}
\end{equation*}

$\therefore |A^n\cap B^n|\geqslant (1-\epsilon _0)2^{n(H-\epsilon )}=\frac{1}{2}2^{n(H-\epsilon )}$
\subsection*{3.11•定理3.3.1证明}

(a)
\begin{equation*}
\begin{split}
Pr(A\cap B)&=Pr(A)+Pr(B)-Pr(A\cup B)\\
&>1-\epsilon _1+1-\epsilon _2 -Pr(A\cup B)\\
&=1-\epsilon _1-\epsilon_2+(1-Pr(A\cup B))\\
&\geqslant 1-\epsilon _1-\epsilon_2
\end{split}
\end{equation*}

(b)由(a)$$1-\epsilon -\delta\leqslant Pr(A_\epsilon^{(n)}\cap B_\delta^{(n)})$$

由概率函数定义
\begin{equation*}
\begin{split}
Pr(A_\epsilon^{(n)}\cap B_\delta^{(n)})=\sum _{A_\epsilon^{(n)}\cap B_\delta^{(n)}}p(x^n)
\end{split}
\end{equation*}

由$2^{-n(H+\epsilon)}\leqslant p(x^n)\leqslant 2^{-n(H-\epsilon)}$,

\begin{equation*}
\sum _{A_\epsilon^{(n)}\cap B_\delta^{(n)}}p(x^n)\leqslant \sum _{A_\epsilon^{(n)}\cap B_\delta^{(n)}}2^{-n(H-\epsilon)}
\end{equation*}

$A_\epsilon^{(n)}\cap B_\delta^{(n)}$中共包含元素$|A_\epsilon^{(n)}\cap B_\delta^{(n)}|$个,$ 2^{-n(H-\epsilon)}$为常数,故
\begin{equation*}
\sum _{A_\epsilon^{(n)}\cap B_\delta^{(n)}}2^{-n(H-\epsilon)}=|A_\epsilon^{(n)}\cap B_\delta^{(n)}|2^{-n(H-\epsilon)}
\end{equation*}

由集合性质,$A_\epsilon^{(n)}\cap B_\delta^{(n)}\subset B_\delta^{(n)} $,则$|A_\epsilon^{(n)}\cap B_\delta^{(n)}|\leqslant |B_\delta^{(n)}| $,则

\begin{equation*}
|A_\epsilon^{(n)}\cap B_\delta^{(n)}|2^{-n(H-\epsilon)}\leqslant |B_\delta^{(n)}|2^{-n(H-\epsilon)}
\end{equation*}

(c)定理3.3.1: 设$X_1,X_2,\cdots ,X_n$为服从p(x)的i.i.d序列。对$\delta < \frac{1}{2}$及任意的$\delta ^\prime >0$,如果$Pr\{B_\delta^{(n)}\}$,则对于充分大的n$$\frac{1}{n}log|B_\delta^{(n)}|>H-\delta^\prime$$

证明:由(b)$$1-\epsilon -\delta\leqslant |B_\delta^{(n)}|2^{-n(H-\epsilon)}$$

已知$\epsilon <\frac{1}{2}, \delta <\frac{1}{2}\Rightarrow 0<1-\epsilon-\delta <1$
$$\therefore\frac{1}{n}log|B_\delta^{(n)}|\geqslant H-\epsilon +\frac{log(1-\epsilon-\delta) }{n} $$

$(\epsilon -\frac{log(1-\epsilon-\delta)}{n})\in(\epsilon,+\infty )$,则对于任意$\delta^\prime>0$,存在$0<\epsilon<\delta^\prime$使$\epsilon -\frac{log(1-\epsilon-\delta)}{n}=\delta ^\prime $,原命题成立。
\end{document}
