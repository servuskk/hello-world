%用来打印中文
\documentclass[UTF8]{ctexart}
\RequirePackage{iftex}
\RequirePackage{fix-cm}
\RequirePackage{fixltx2e}
%页面设置
\RequirePackage{geometry}
\if@twoside
 \geometry{a4paper,
 bindingoffset = 2cm,
 inner = 0.5cm,
 outer = 2cm,
 top = 3cm,
 bottom = 2cm
 }
\else
 \geometry{a4paper,
 left = 2cm,
 right = 2cm,
 top = 2cm,
 bottom = 2cm
 }
%各种包
\usepackage{fancyhdr}
\RequirePackage{graphicx}
\RequirePackage{subfigure}
\RequirePackage{caption}
\RequirePackage{diagbox}
\RequirePackage{multirow}
\RequirePackage{makecell}
\RequirePackage{booktabs}
\usepackage{listings}%代码
%矩阵
\usepackage{amsmath,xcolor}
\RequirePackage{longtable}
\RequirePackage{array}
%页眉
\RequirePackage{float}
\RequirePackage{flowchart}
\pagestyle{fancy}
\renewcommand{\headrulewidth}{0.5pt} 
\lhead{} \rhead{此处是页眉}
%如果需要有标题页,把以下两行放到\begin{document}之后,写作业的话大可不必
%\title{标题}
%\maketitle%显示标题和日期
\begin{document}  
\section{一级标题}
\subsection{二级标题}
本人是一名练习时长8个月的Latex菜鸡用户,网课期间针对需要用latex提交作业的科目写了这样一篇小小的汇总,希望可以帮助大家节约排版时间~
\textbf{(本文以完成作业为最高目标做了一个常用汇总,想到哪写到哪,不全面且无逻辑)}

所以,用过Latex的小伙伴就不需要看了……
\subsubsection{三级标题}
一些基本的坑:
\begin{enumerate}
\item[•]本文内容代码和PDF同时打开食用效果更佳!自认为代码里的注释还比较详细?
\item[•]最前面的那一堆(看代码)一定要有,否则不能编译通过。如果能找到合适的,找一个模板吧,上面那一堆就可以直接换成一行调用模板了。
\item[•]写公式一定要用\$ 或者displaymath等,不能直接写在正文里;如果公式或数学符号出现在一句话里且不需要单独一行,前后分别使用一个\$,如果需要换行,就是公式或字母单独一行,前后分别要用两个\$.
\item[•] 你看到的换行不是真正的换行,真正的换行是空一行。
\item[•]如果需要显示Latex 的保留字符,比如注释的百分号,要这样写:\%
\item[•]空格也要这样写\ ,如果要空比较大的格\quad 用quad。
\item[•]使用\textbf{XELATEX}编译!!!!!!
\end{enumerate}
\subsection{常用汇总}
\textbf{下标}用下划线,如果下标不止一位,要使用\{ \}
$$t_0$$
$$t_{10}$$

\textbf{上标}用\^,不止一位用\{\}
$$e^x$$
$$e^{2x}$$


\textbf{向量}$\vec{B}$

%矩阵
\subsection{矩阵:} 
\begin{displaymath}
\vec{B}\times\vec{C}= \left[
\begin{matrix}
   \vec{a}_x & \vec{a}_y & \vec{a}_z \\
   0 & -4 & 1 \\
   5 & 0 & -2
\end{matrix}
\right] =8\vec{a}_x+5\vec{a}_y+20\vec{a}_z
\end{displaymath}

%使用ref引用文中的图表会自动编号
\subsection{表格:}
如表\ref{tab:引用名称}:
\begin{table}[H]%表格
    \centering
    \caption{显示在论文里的表格名}
    \label{tab:引用名称}
    \begin{tabular}{c c}%设置列数
\hline
    	Symbol & Definition\\
\hline
	$N$ & total population\\
	$S(t)$ & number of susceptible people at time $t$\\
\hline
    \end{tabular}
\end{table}

\subsection{希腊字母:} 
如表\ref{tab:希腊字母}:
\begin{table}[H]
\centering
\caption{希腊字母}
\label{tab:希腊字母}
\begin{tabular}{|c c c c|}
\hline
$\alpha$ &$\beta$ &$\gamma$ &$\delta$ \\
$\epsilon$ &$\varepsilon $ &$\zeta$ &$\eta$ \\
$\theta$ &$\vartheta$ &$\iota$ &$\kappa$ \\
$\lambda$ &$\mu$ &$\nu$ &$\xi$ \\
$\pi$ &$\varpi$ &$\rho$ &$\varrho$ \\
$\sigma$ &$\varsigma$ &$\tau$ &$\upsilon$\\
$\phi$ &$\varphi$ &$\chi$ &$\psi$\\
$\omega$ &$\Gamma$ &$\Delta$ &$\Theta$\\
$\Lambda$ &$\Xi$ &$\Pi $&$\Sigma$\\
$\Upsilon$ &$\Phi $&$\Psi$ &$\Omega$\\
\hline
\end{tabular}
\end{table}

\subsection{分数:} 
$\dfrac{A}{B}$

\subsection{方程组:} 
带花括号的方程组:
%displaymath里不能有空行
\begin{displaymath}
v_{(t)}=\begin{cases}
\dfrac{N_{0}-N_{1}}{t_{1}},&0<t<t_{1}\cr
C_1e^{C_2-C_1t},&t>t_1\end{cases}
\end{displaymath}

带编号的方程:
\begin{equation}
a+b=c
\end{equation}
\subsection{积分:} 
$$\displaystyle{\int_{t_0}^{t_0+T(t_0)}} v_{(t)}dt = \displaystyle {\int_{t_0}^{t_0+T(t_0)}} \dfrac{N+Q(t_0)+ \theta N(t_0)}{T(t_0)}dt$$
\subsection{求和求导极限:} 
这里有个整理的比较好的:https://www.cnblogs.com/liangjianli/p/11616847.html
这个博客整理的很好,加上这个PDF里写作业应该就足够了!如果要写论文建议去找个模板,省时省力。
\subsection{列举:} 
列举1(这个没有编号):
\begin{enumerate}
\item[第一条]前面的序号可以自己指定,不会自动编号
\item[第二条]
\end{enumerate}

列举2(自动编号):
(我写作业一般用这个,把题干加粗看起来就8错)
\begin{enumerate}
\item 第一条
\item 第二条
\item \textbf{电子计算机一般分成哪些组成部分?为什么要分成这些组成部分?}
%换行要空一行!(写了注释就不算空行了,所以这里又空了一行)

运算器、存储器、控制器、输入单元、输出单元。

\end{enumerate}


\subsection{figures and tables}
\begin{figure}[h]
    \centering
      \includegraphics[width=10cm]{/Users/whh/Documents/MATLAB/random/question3-0.png} 
      \caption{单张图片} 
     \label{fig:q3-0}
  \end{figure}

\begin{figure}[!htp]
    \centering
    \begin{subfigure}
      \centering
      \includegraphics[height=6cm]{/Users/whh/Documents/MATLAB/random/question2-5(v3).png}
    \end{subfigure}
    \hspace{0.5cm}
    \begin{subfigure}
      \centering
      \includegraphics[height=6cm]{/Users/whh/Documents/MATLAB/random/question2-6(v4).png}
    \end{subfigure}
    \caption{两张图并列}
    \label{fig:q2-v2}
  \end{figure}

  \begin{minipage}{\textwidth}

    \begin{minipage}[t]{0.48\textwidth}
        \centering
    \makeatletter\def\@captype{table}
    \begin{tabular}{cc}
        \toprule
        参数&值\\
        \hline
        $\alpha$& $ 1$\\
        $v   $  &$   10$   \\
        $\sigma$  &$  1 $   \\
        $x_0   $  &$ 10  $   \\
        \bottomrule
    \end{tabular}
    \caption{Monte Caro方法计算$E(X_1),D(X_1)$参数}
    \label{tab:q2}
    \end{minipage}
    \begin{minipage}[t]{0.48\textwidth}
        \centering
    \makeatletter\def\@captype{table}
    \begin{tabular}{ccc}
        \toprule
        序号&$EX_1$&$DX_1$\\
        \hline
       1&10.0059  &  0.4376\\
        2&10.0132  &  0.4335\\
        3&9.9931    &0.4296\\
        \hline
        AVG&10.0041&0.4455\\
        \bottomrule
    \end{tabular}
    \caption{$E(X_1),D(X_1)$计算结果}
    \label{tab:q2-2}
    \end{minipage}
    \end{minipage}

\newpage
\appendix
\textbf{附录}

\section{用来贴代码,当然格式有好多种,不列举了}
\begin{lstlisting}
%直接复制法
mu1=0;
sigma1=25;
mu2=100;
sigma2=1;
X=normrnd(mu1,sigma1,[1 10000]);
Y=normrnd(mu2,sigma2,[1 10000]);
n=unifrnd(0,1,1,10000);
p=0.3;
n(n>p)=0;
n(n>0)=1;

Z=X+n.*Y;
[counts,centers] = hist(Z, 300);
figure
bar(centers, counts / sum(counts))
title("Figure1")
\end{lstlisting}
\section{另一种:使用文件}
\lstinputlisting[language=Matlab]{go.m}
\section{写在最后}
看到这里,应该就算是可以正常使用latex了,如果在使用没有提到过的功能,直接上网搜就行,如果出现报错,有几个建议:
\begin{enumerate}
\item 检查是否安装了所使用的结构的库
\item 检查是否有括号或者\$没有补全
\item 参考文献使用.bib文件时,编译的顺序时:XELatex->bibtex->XElatex(都是编译.tex文件,不需要编译.bib!!!!!)
\item 还有一个我之前整理的半成品,有兴趣可以看看:https://shimo.im/docs/qxPwK33KwghrtjJt/ 
《latex常用汇总》,可复制链接后用石墨文档 App 或小程序打开
\item \textbf{奇淫巧计:}如果有什么不会用latex解决或者觉得用latex太麻烦的问题,可以尝试一下编辑PDF(Adobe Acrobat DC大法好)
\item \textbf{IEEE论文写作及作业示例} IEEE英文模板见官网,作业示例见文件夹examples
\end{enumerate}
emm好了,结束了。

第一次编辑:2020/03/14
第二次编辑:2022/12/23
\end{document}  
